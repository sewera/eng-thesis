\subsection{Git hooks}\label{sec:git-hooks}

Git hooks are a great tool
for enforcing common style guidelines,
commit message format,
and as a general reminder
to perform certain tasks before checking in the code.
They are shell scripts ran by Git
when certain events occur.
I use pre-commit and commit-msg hooks.
A pre-commit hook runs just before committing
and prevents a commit if it fails.
A commit-msg hook simply checks
if a commit message written by a programmer
meets certain criteria.

From my professional experience,
git hooks have to be very quick,
preferably almost instant,
otherwise they will be skipped
with a \texttt{{-}{-}no-verify} flag.
At the beginning,
I have set up git hooks
with Husky~\cite{typicode_husky_2022},
but being written in \ac{JS},
they sometimes ran for over half a minute.
I decided to go with manually written shell scripts
as git hooks.
They turned out to be very performant,
easy to set up, and customize.
For a pre-commit hook I set up code formatting checks.
If the staged for commit code is not formatted correctly,
it fails.
For a commit-msg hook I set up a regular-expression-based
message checking.
The commit message has to follow
Conventional Commits Specification~\cite{petrungaro_conventional_2019}.
