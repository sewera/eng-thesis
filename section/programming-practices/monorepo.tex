\subsection{Monorepo}\label{sec:monorepo}

Monorepo is just a single repository
for the whole solution.
All the components are in one repository,
including frontend,
backend,
producer, and
testing tools.
It greatly improved and simplified
the integration between frontend, backend,
and gave way for scripts
bringing the whole stack up.
The unit and integration tests run for all
components simultaneously,
so that if something in the product breaks,
it is immediately visible.
I can take full advantage of the
``stop the presses'' \ac{CI} pipeline failure approach,
explained in \citetitle{martin_clean_2011}~\cite{martin_clean_2011}.

Not only that,
but it also simplified sharing
the common protocol models (section~\ref{sec:protocol})
for the Go projects, i.e. backend and producer,
without the need for additional procedures
for releasing the libraries with models.

Such monorepos provide many advantages
when keeping the whole product in one place.
Refactoring,
even a full-stack-wide
is very easy in such a repository,
and can be brought down to a single \ac{PR}.
This in turn decreases the lag between
opening a \ac{PR} and merging it,
compared to many \acp{PR} in many different repositories.
It is especially important
to optimize the \ac{PR} times in open-source,
because here we cannot avoid them,
in contrast to the \ac{XP} teams~\cite{beck_extreme_2004},
in which the only contributors are its members.
