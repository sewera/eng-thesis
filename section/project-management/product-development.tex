\subsection{Product development}\label{sec:product-development}

The most important thing I learned from project management was undoubtedly
how to reliably conduct product development.
When I started developing Notipie,
I had a very vague vision of what the finish product will provide.
The essential functionality was interleaved with features
that could not be implemented without the essentials.

That is when I decided to streamline the development
and set a constraint of what needs to be done
in order to consider the product usable.
I defined the \textbf{Minimally Viable Product}
and prioritized the issues~\cite{sewera_issues_2022}.

\subsubsection{Minimally Viable Product}\label{sec:minimally-viable-product}

The definition is as follows~\cite{sewera_mvp_2022}:

\begin{quote}
      Notifications can be sent with a manually programmed script
      with the help of a library.
      They arrive to the user interface in real time.

      The library is provided in the MVP.\\
      The backend correctly forwards those notifications to the frontend.\\
      The backend stores existing notifications in memory.\\
      The frontend correctly presents notifications to the user.
\end{quote}
