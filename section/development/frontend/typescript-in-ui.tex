\subsubsection{TypeScript in UI}\label{sec:typescript-in-ui}

I decided to use \acl{TS} in my project for the frontend part,
because of its type checking tools,
huge popularity,
and a growing demand for in on the job market.

\paragraph{Choosing the language}\label{sec:choosing-the-language}

When choosing which language to use in the \ac{UI},
I considered a couple of options:

\begin{itemize}
  \item
        plain \acl{JS},
  \item
        \acl{TS},
  \item
        Elm, and
  \item
        CoffeeScript.
\end{itemize}

I immediately discarded the last two,
due to their smaller popularity,
compared to \acl{JS} or \acl{TS}.
Elm and CoffeeScript may have seen
a spike of popularity before \acl{TS}
became main stream,
but as of now, they have a minuscule market share.

The feature set of the language was also very important to me.
\Acl{JS} is by far the most popular,
but it lacks type annotations or pre-runtime type checking.
\Acl{TS} and Elm turned out to be winners in the type checking toolchain.
\Acl{TS} also has a big advantage of being very similar to plain \acl{JS},
so the transpiled code is very readable.

A big factor was general trend of language's popularity growth.
\Acl{TS} was a clear winner in this scenario,
being third most loved language
and second most wanted language
in the Stack Overflow Developer Survey 2021~\cite{stack_overflow_2021_2021}.
It was only beaten by Rust and Clojure
in the \textit{Most Loved} section,
both of which are non-frontend languages,
and Python in the \textit{Most Wanted} section,
which is also not a frontend language.
Another report confirming the growing popularity of \acl{TS}
is Github Octoverse Report 2021~\cite{github_inc_2021_2021}.
Since 2017,
it beat
Ruby,
C,
C++,
C\#,
Shell, and
\ac{PHP}
and is, as of 2021, the fourth top language on Github.

\paragraph{Working with TypeScript}\label{sec:working-with-typescript}

Starting with \acl{TS} was fairly easy.
The toolchain was included in the project creation scripts.
Most dependencies had good \acl{TS} annotations,
or they were completely written in \acl{TS},
which was very helpful for maintaining type safety.

Learning the language was also very easy.
I was already familiar with \acl{JS},
so I only needed to learn the type annotations,
which were very intuitive to use.
I also found out that because of the type annotations,
function composition became much easier,
so I had less problem working with more complex
language structures.
\Acl{TS} made writing code easier than \acl{JS}.
