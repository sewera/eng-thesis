\subsubsection{Build system}\label{sec:build-system}

For the build and bundle software,
I wanted to use something quite modern,
with hot module reloading,
easy to use setup scripts,
customizable development server,
and short bundle times.
One library I immediately ruled out
because of lack of those modern features
was Webpack~\cite{koppers_webpack_2022}.
It is a tool with a lot of legacy,
which brought \ac{JS} bundling to the main stream.
However, today its legacy gets in the way
of a clean \ac{API} and good developer experience,
compared to modern tools.

\paragraph*{Snowpack and Vite}\label{sec:snowpack-and-vite}

I started with Snowpack~\cite{schott_snowpack_2021}
and used it until I decided to move to Vite~\cite{you_vite_2022}
in commit \texttt{c11bc35}\footfullcite{sewera_choreui_2021}.
Snowpack offered both hot module reloading and short bundle times.
Nevertheless,
there were some minor glitches and bugs from time to time,
the project had a slow development,
small user base,
and the alternative, Vite,
did not seem to have those problems.

I tried Vite in my other project,
Reactive Raven~\cite{sewera_reactive_2022},
and the integration with
React,
\ac{TS},
Tailwind~CSS,
and other tools I used was seamless,
therefore I decided to migrate to it in Notipie as well.
On April 20th, 2022,
Snowpack's maintainer stated in the project's Readme document
(commit \texttt{45456aa}\footfullcite{schott_readmemd_2022})
that he would no longer maintain the project,
and mentioned Vite as a good alternative for it.

\addtocategory{commit}{sewera_choreui_2021,schott_readmemd_2022}
