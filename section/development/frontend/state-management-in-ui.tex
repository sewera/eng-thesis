\subsubsection{State management in UI}\label{sec:state-management-in-ui}

To simplify the frontend code,
I needed to use a single source of truth for the data.
Store,
which is a solution for state management
in frontend applications,
is~a~Singleton~\cite{gamma_design_1994}
that can be read by any component,
but can be mutated only using certain functions,
or, in Redux library's case, dispatching actions.
This~approach prevents any data races
or having the store in an unstable state.
I~used both Redux~\cite{gaeraon_redux_2022},
and Zustand~\cite{kato_zustand_2022} for this task
as store implementations,
and Zustand came on top as a simpler solution for my application.

\paragraph*{Redux}\label{sec:redux}

Redux is great for big applications with lots of components.
Being one of the most popular state management libraries for React,
it was my first choice.
Unfortunately,
it required me to write a lot of boilerplate code,
and thus was not easily maintainable
for a smaller project like Notipie.

\paragraph*{Zustand}\label{sec:zustand}

Zustand is a lot simpler than Redux,
requires a lot less boilerplate code,
and was sufficient for my application.
I migrated to it in commit
\texttt{7677d13}\footfullcite{sewera_choreui_2022},
and it reduced the lines of code by over 200.
I did not, however, give up the connected components,
as they provide better testability and separation of concerns,
which is worth a bit extra code.

\addtocategory{commit}{sewera_choreui_2022}
