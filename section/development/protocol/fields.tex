\subsubsection{Fields}\label{sec:protocol-fields}

I created two versions of notification models.
One for the App,
and one for the client\footnote{
  A client is, e.g.,
  a browser tab running Notipie \ac{UI},
  and is represented
  by a User in the domain (section~\ref{sec:user}).
}.
They have a common subset of fields,
named \texttt{HashableNetNotification}.
\textit{Hashable},
because this portion is hashed with 256-bit \ac{SHA-2}
in order to generate a unique notification \ac{ID}.
The following fields are common:

\begin{itemize}
  \item Timestamp -- time when the notification was sent
        in RFC3339 format\footnote{
          RFC3339~\cite{clyne_rfc3339_2002} is based on
          ISO 8601:2000~\cite{international_organization_for_standardization_iso_2000},
          published by \ac{ISO} in 2000,
          but even the latest, 2019 version of ISO 8601
          is compatible with RFC3339 in this use case.
        }, required,
  \item AppName -- the name of the App sending notification, required,
  \item AppID -- the \ac{ID} of the App
        if it sends a notification after the first one\footnote{
          This mechanism is described in section~\ref{sec:app}.
        },
  \item AppImgURI -- the image \ac{URI} of the app
        used as a logo on a notification card;
        if empty, a logo is generated from the AppName,
  \item Title -- the title of a notification, required,
  \item Subtitle -- an optional subtitle of a notification,
  \item Body -- an optional description of a notification,
  \item ExtURI, ReadURI, and ArchiveURI -- \acp{URI}
        for seeing the notification in an external service,
        marking it as read in this service, and archiving it.
\end{itemize}

The notifications are further specified
for an App, and a client.
For an App, it is called \texttt{AppNotification},
and for a client, \texttt{ClientNotification}.
Both App and client versions
have an additional common field, \ac{ID},
which is a notification \ac{ID}
based on a hash from \texttt{HashableNetNotification}.
It could not be included in the common portion,
because the hash of such common set of fields
would change after putting in the value of the hash.
App version has an additional field of ApiKey,
and client version has a Read indicator
to mark if the notifications received
synchronously are already read.
