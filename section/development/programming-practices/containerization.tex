\subsection{Containerization}\label{sec:containerization}

% TODO: Elaborate (eng-thesis #22)
Containerization is a technique of running an application
in a virtualized environment.
A container provides all the necessary dependencies
for the application,
for instance, shared libraries,
like \texttt{libc} and core utils implementations.
There is a difference between container-based virtualization
(containerization) and full virtualization, though.
The container runs on a shared kernel,
as opposed to having its own
virtual CPU, memory, and own kernel~\cite{watada_emerging_2019}.
The main advantage is to maintain separation
and minimize coupling between
the major components of the architecture~\cite{stytz_rapid_1997},
most often microservices.

I decided to use the Docker technology
for containerization in my project.
Not only is it a good fit for microservices,
but also it features many time-savers in its ecosystem.
For example,
defining containers
is as easy as writing a Dockerfile.
The public Docker Registry,
called Docker Hub contains a plethora
of useful existing images ready for use~\cite{jaramillo_leveraging_2016}
that include, inter alia:
Linux distributions, like Debian,
reverse proxies, like Nginx,
or build environments, like Node.js.

To simplify the deployment process,
I decided to containerize my application
with Docker.
For the backend container,
I used different images
for building the binaries
and for running the service.
For the frontend container the approach was similar,
but for generating static files
and hosting them.
The setup is depicted in appendix~\ref{apx:containerization-with-docker}.
