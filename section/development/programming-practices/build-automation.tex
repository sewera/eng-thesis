\subsubsection{Build automation}\label{sec:build-automation}

A build system is a program or set of programs
that automates commonly-used actions.
It dramatically speeds up the development cycle,
reduces manual repetition of the tasks,
and in turn,
reduces the number of possible mistakes
that could be made had those tasks
been performed by hand.
Those actions include:
\begin{itemize}
  \item dependency synchronization,
  \item project compilation,
  \item project testing,
  \item starting a development environment,
  \item creating Docker images,
  \item and more.
\end{itemize}

When developing the application,
it is very important to automate the build process.
For this task,
I chose \texttt{package.json} scripts
for \ac{TS} projects,
and Make for everything else.

It was very important for me
to have this project ready for development
after running only one script.
That is why I prepared a Make recipe
which guides a user on how to install
Go, Node.js, and Yarn,
which installation could not be automated,
and installs the prerequisites for all components.
It also copies example configuration files
to the necessary places,
and installs Git hooks,
so that a full local development setup
is configured.

I also provided many recipes
for configuration files management,
building binaries,
Docker images creation,
quick application running with hot reload,
testing,
linting, and
code formatting
to further speed up the tasks
that developers do repeatedly.

I chose to use \texttt{package.json} scripts
for \ac{TS} projects,
because they are widely used
across the frontend developers,
and there is a better chance
that when someone wants to contribute
only to the frontend project,
they will be more familiar with those scripts
than with Make.
