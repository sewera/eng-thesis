\subsection{The benefits of using Go and its standard library}\label{sec:the-benefits-of-using-go-and-its-standard-library}

Go is an excellent language for writing microservices.
Its focus on this one task
and pragmatism in adding features to the language by its authors,
resulted in an easy to understand and use,
yet very powerful set of tools.

\subsubsection{Goroutines and channels}\label{sec:goroutines-and-channels}

Among of the best features of the Go language lay goroutines and channels.
They make concurrent programming a lot easier, compared to other languages.
I used both goroutines and channels
for inter-object communication in \texttt{domain} package.

For example, in {tag.go}~(appendix~\ref{apx:concurrency-in-go}),
after the Tag object is created,
the constructor calls
the \texttt{start} method~(listing~\ref{lst:start-method-in-tag}).
It is running a new goroutine for every Tag instance,
enabling them to asynchronously communicate with other objects.

The Tag is actively listening
on the Notification channel (\texttt{NotificationChan}),
and every time a new Notification arrives,
the \texttt{broadcast} method~(listing~\ref{lst:broadcast-method-in-tag})
is executed.

The \texttt{broadcast} method in turn sends that Notification
to every User subscribed to this Tag.
The flow is presented in figure~\ref{fig:notification-flowchart}.

\begin{figure}[h]
  \centering
  \includegraphics[width=10cm,keepaspectratio]{chart/out/notification-flowchart.pdf}
  \caption{Notification flow in \texttt{domain}}
  \label{fig:notification-flowchart}
\end{figure}

There is an obvious problem with this approach.
When a User is subscribed to two Tags,
and one App has those two Tags assigned,
then that User can get the same Notification twice
(figure~\ref{fig:duplicated-notification}).

\begin{figure}[h]
  \centering
  \includegraphics[width=10cm,keepaspectratio]{chart/out/duplicated-notification.pdf}
  \caption{Duplicated Notification flow}
  \label{fig:duplicated-notification}
\end{figure}

There are two solutions I could think of.

\paragraph*{Solution 1}

There is a new component in the domain, Router.
It is a centralized component,
that knows which App is connected to which Tag,
and which User is subscribed to those Tags.

// TODO: finish (notes from e-book)

\subsubsection{Refactoring to the standards}\label{sec:refactoring-to-the-standards}

During this project,
I refactored my functions to better suit
the Go standard library.

For instance,
in commit \texttt{00547cd}\footfullcite{sewera_chorecoreproducer_2022},
I refactored the \texttt{ToJSON} method
to better suit the standard signatures
(appendix~\ref{apx:method-signature-refactoring-in-go}),
i.e. to return the byte array and error,
just like the \texttt{Marshal} method
from the \texttt{json} package~\cite{cox_json_2022}
in the standard library.

\addtocategory{commit}{sewera_chorecoreproducer_2022}
