\subsubsection{Core project architecture}\label{sec:core-project-architecture}

An overview of the backend architecture
from the asynchronous notification flow perspective
is presented in figure~\ref{fig:high-level-backend-flow}.
The project on the top level is structured over 4 directories,
named according to
the \citetitle{quest_standard_2022}~\cite{quest_standard_2022}:

\begin{figure}[h]
      \centering
      \includegraphics[width=6.5cm,keepaspectratio]{chart/out/backend-flow.pdf}
      \caption{Backend architecture for asynchronous notification flow}
      \label{fig:high-level-backend-flow}
\end{figure}

\begin{itemize}
      \item
            \texttt{cmd} -- entry point to the application (\texttt{main}),
      \item
            \texttt{internal} -- application-specific code,
      \item
            \texttt{pkg} -- reusable utils, not specific to the application,
      \item
            \texttt{test} -- black box integration test code.
\end{itemize}

\paragraph*{The internal directory}\label{sec:the-internal-directory}

Application-specific code is split into 5 directories
representing the levels of abstraction:

\begin{itemize}
      \item
            \texttt{domain} -- business logic,
            defines data structures and communication of domain objects
            on the highest level of abstraction,
      \item
            \texttt{grid} -- lower level of abstraction than domain,
            defines proxies that convert network models into domain models,
            and connects those proxies with the domain components,
      \item
            \texttt{impl} -- implements network endpoints,
            \ac{WS} connections, and persistence,
      \item
            \texttt{infra} -- configures the application
            and sets up the context for \ac{DI}.
\end{itemize}

There also was a \texttt{model} package
in the \texttt{internal} directory,
but I moved it to an exportable \texttt{pkg} directory
in order to reuse it in the notification producer.
This package includes reusable network models,
used for communication between application components,
like backend and producer.
More on the communication protocol in section~\ref{sec:protocol}.

\paragraph*{Grid}\label{sec:grid}

The grid is an application layer
that connects implementations of
the endpoints, \ac{WS} connections, and persistence
with the domain objects.
I decided to define that name in the \ac{UL}
to mean this exact intermediate layer.
The name itself was inspired by a power grid
which connects components
from power generation
to appliances.

The implementation is based on the proxies
for domain Apps and Users\footnote{
      The App and the User are defined in
      sections~\ref{sec:app}~and~\ref{sec:user}
      respectively.
}.
Those proxies convert the applicable network models
and perform intermediate procedures
that do not concern the domain,
like Notification \ac{ID} generation.
This layer is especially important
to keep the domain clean and concise,
and to enable easy replacing\footnote{
      The differences between writing code
      for reuse and replacement are described in detail
      in section~\ref{sec:code-quality}.
}
of concrete implementations.
