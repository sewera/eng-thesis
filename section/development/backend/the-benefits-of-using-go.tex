\subsubsection{The benefits of using Go}\label{sec:the-benefits-of-using-go}

\paragraph*{Built-in language features}\label{sec:built-in-language-features}

There were numerous features
that helped during backend development.
Implicit encapsulation of functions and variables
starting with a lowercase letter,
structure field annotations
enabling easy serialization
to \ac{JSON} and \ac{YAML},
coroutines automatically managed by the Go runtime,
or channels,
just to name a few.
The idea behind most of them
was very simple to understand,
the usage was intuitive,
and I did not have to resort to the documentation that often.

\paragraph*{Goroutines and channels}\label{sec:goroutines-and-channels}

Among of the best features of the Go language lay goroutines and channels.
They make concurrent programming a lot easier, compared to other languages.
I used both goroutines and channels
for inter-object communication in \texttt{domain} package.

For example, in {tag.go}~(appendix~\ref{apx:concurrency-in-go}),
after the Tag object is created,
the constructor calls
the \texttt{start} method~(listing~\ref{lst:start-method-in-tag}).
It is running a new goroutine for every Tag instance,
enabling them to asynchronously communicate with other objects.

The Tag was a particularly special case,
in which I had to solve
the Notification duplication problem,
described in detail in section~\ref{sec:tag-technicalities}

\paragraph*{Standard library}\label{sec:standard-library}

Standard \texttt{testing} package~\cite{cox_testing_2022} provided
a unified, and simple tooling for testing.
I did not have to think anything about test setup.
No custom scripts, third-party libraries, or \ac{IDE} setup.
All I needed to do was to name a file with a \texttt{\_test.go} suffix,
write a function starting with \texttt{Test},
and run \texttt{go\ test\ ./...}.
Both \ac{VSCode} with Go plugin and GoLand
automatically picked up the test setup,
and I was ready to develop with \ac{TDD}.

Standard \texttt{net/http} package~\cite{cox_http_2022} provides everything
needed for setting up \ac{REST} endpoints.
Although I used Gin~\cite{martinez-almeida_gin_2022} for this,
due to a simpler interface,
I used status codes
and \ac{HTTP} client implementation
from \texttt{net/http}.

\paragraph*{Refactoring to the standards}\label{sec:refactoring-to-the-standards}

During this project,
I refactored my functions to better suit
the Go standard library.
It was beneficial
not only because of a better integration
with the standard library itself,
but also because of a better integration
with third-party libraries.
There is an unwritten rule,
that every library strives
to be as compatible
with the standard library as possible.

For instance,
in commit \texttt{00547cd}\footfullcite{sewera_chorecoreproducer_2022},
I refactored the \texttt{ToJSON} method
to better suit the standard signatures
(appendix~\ref{apx:method-signature-refactoring-in-go}),
i.e. to return the byte array and error,
just like the \texttt{Marshal} method
from the \texttt{json} package~\cite{cox_json_2022}
in the standard library.

\paragraph*{Third-party libraries}\label{sec:third-party-libraries}

Gin was great for writing \ac{REST} endpoints,
with \texttt{gin.Con\-text} having easy access to
standard-library-compatible fields,
making it easily pluggable to other third-party libraries,
like Gorilla WebSocket~\cite{burd_gorilla_2022}.

Zap~\cite{shah_zap_2022} provided a reliable
and performant way to log things in the backend.
Structured logging,
straightforward syntax,
automatic serialization to \ac{JSON} in production mode,
and human-readable format in debug mode,
paired with low or zero-allocation overhead,
made it a perfect choice for logging in a microservice.

\addtocategory{commit}{sewera_chorecoreproducer_2022}
