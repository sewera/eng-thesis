\subsection{Backend}\label{sec:backend}

The backend is a component in a system
that is responsible for the data processing.
It interacts with the frontend applications
with a well-known protocol,
usually one understood by a browser.

The backend of Notipie
needed to be performant,
easy to deploy,
and be of~good quality,
which means it has to be well-tested.
I strived to make use of the best practices
in the field of microservices,
maintain great quality,
and ensure sufficient performance.
I named it \textit{core},
because there can be different \acp{UI},
or notification producers,
but the core of Notipie
is specifically the backend implementation.
The backend code can be found in the
\texttt{core} directory of the Notipie project~\cite{sewera_notipie_2022}.

\begin{figure}[!ht]
  \centering
  \includegraphics[width=5cm,keepaspectratio]{chart/out/backend-flow.pdf}
  \caption{Backend architecture for asynchronous notification flow}
  \label{fig:high-level-backend-flow}
\end{figure}

\subsubsection{Core project architecture}\label{sec:core-project-architecture}

An overview of the backend architecture
from the asynchronous notification flow perspective
is presented in figure~\ref{fig:high-level-backend-flow}.
The project on the top level is structured over 4 directories,
named according to
the \citetitle{quest_standard_2022}~\cite{quest_standard_2022}:

\begin{figure}[h]
      \centering
      \includegraphics[width=6.5cm,keepaspectratio]{chart/out/backend-flow.pdf}
      \caption{Backend architecture for asynchronous notification flow}
      \label{fig:high-level-backend-flow}
\end{figure}

\begin{itemize}
      \item
            \texttt{cmd} -- entry point to the application (\texttt{main}),
      \item
            \texttt{internal} -- application-specific code,
      \item
            \texttt{pkg} -- reusable utils, not specific to the application,
      \item
            \texttt{test} -- black box integration test code.
\end{itemize}

\paragraph*{The internal directory}\label{sec:the-internal-directory}

Application-specific code is split into 5 directories
representing the levels of abstraction:

\begin{itemize}
      \item
            \texttt{domain} -- business logic,
            defines data structures and communication of domain objects
            on the highest level of abstraction,
      \item
            \texttt{grid} -- lower level of abstraction than domain,
            defines proxies that convert network models into domain models,
            and connects those proxies with the domain components,
      \item
            \texttt{impl} -- implements network endpoints,
            \ac{WS} connections, and persistence,
      \item
            \texttt{infra} -- configures the application
            and sets up the context for \ac{DI}.
\end{itemize}

There also was a \texttt{model} package
in the \texttt{internal} directory,
but I moved it to an exportable \texttt{pkg} directory
in order to reuse it in the notification producer.
This package includes reusable network models,
used for communication between application components,
like backend and producer.
More on the communication protocol in section~\ref{sec:protocol}.

\paragraph*{Grid}\label{sec:grid}

The grid is an application layer
that connects implementations of
the endpoints, \ac{WS} connections, and persistence
with the domain objects.
I decided to define that name in the \ac{UL}
to mean this exact intermediate layer.
The name itself was inspired by a power grid
which connects components
from power generation
to appliances.

The implementation is based on the proxies
for domain Apps and Users\footnote{
      The App and the User are defined in
      sections~\ref{sec:app}~and~\ref{sec:user}
      respectively.
}.
Those proxies convert the applicable network models
and perform intermediate procedures
that do not concern the domain,
like Notification \ac{ID} generation.
This layer is especially important
to keep the domain clean and concise,
and to enable easy replacing\footnote{
      The differences between writing code
      for reuse and replacement are described in detail
      in section~\ref{sec:code-quality}.
}
of concrete implementations.

\subsubsection{Go in core}\label{sec:go-in-core}

Go is quickly gaining popularity among developers,
with its great tooling
and a state-of-the-art standard library.
It is an excellent language for writing microservices.
Its focus on this one task
and pragmatism in adding features to the language by its authors,
resulted in an easy to understand and use,
yet very powerful set of tools.

\paragraph{Motivation}\label{sec:motivation}

When choosing the right language for the project,
I focused on finding the right tool
for the application and developer experience.

I wanted \texttt{notipie} to be a high-performance microservice,
so I did not take interpreted languages
like Python or \acf{JS} into consideration.
I mostly considered Java, Kotlin, Rust, and Go.

Java, although popular, does not have the greatest developer experience.
Things like \texttt{equals} and \texttt{hashCode}
are unnecessary bloat in the code.
Project Lombok~\cite{zwitserloot_project_2022} fixes some of them,
but the tooling is limited to IntelliJ,
you have to download a lot of libraries for dealing with \ac{JSON},
create your own code style guide,
and perform a fair bit of setup.

Kotlin, far better than Java,
but also locked-in to IntelliJ with tooling,
was an interesting option for me, but not ideal.

Rust was too low-level for my application.
Explicit memory management, although performant,
was simply too verbose and work-intensive for my use case.

Go was a perfect option.
A plethora of great tooling,
like first-party Go plugin for \ac{VSCode}, GoLand from JetBrains,
community plugins for Neovim,
all working great and providing a good developer experience.
Furthermore, extraordinary performance of the tooling itself,
with tests running in under a second, super-fast compiler,
one of the best standard libraries I have seen,
and overall simplicity of the language, made the choice obvious.

\subsubsection{The benefits of using Go}\label{sec:the-benefits-of-using-go}

\paragraph{Built-in language features}\label{sec:built-in-language-features}

There were numerous features
that helped during backend development.
Implicit encapsulation of functions and variables
starting with a lowercase letter,
structure field annotations
enabling easy serialization
to \ac{JSON} and \ac{YAML},
coroutines automatically managed by the Go runtime,
or channels,
just to name a few.
The idea behind most of them
was very simple to understand,
the usage was intuitive,
and I did not have to resort to the documentation that often.

\paragraph{Goroutines and channels}\label{sec:goroutines-and-channels}

Among of the best features of the Go language lay goroutines and channels.
They make concurrent programming a lot easier, compared to other languages.
I used both goroutines and channels
for inter-object communication in \texttt{domain} package.

For example, in {tag.go}~(appendix~\ref{apx:concurrency-in-go}),
after the Tag object is created,
the constructor calls
the \texttt{start} method~(listing~\ref{lst:start-method-in-tag}).
It is running a new goroutine for every Tag instance,
enabling them to asynchronously communicate with other objects.

The Tag was a particularly special case,
in which I had to solve
the Notification duplication problem,
described in detail in section~\ref{sec:tag-technicalities}

\paragraph{Standard library}\label{sec:standard-library}

Standard \texttt{testing} package~\cite{cox_testing_2022} provided
a unified, and simple tooling for testing.
I did not have to think anything about test setup.
No custom scripts, third-party libraries, or \ac{IDE} setup.
All I needed to do was to name a file with a \texttt{\_test.go} suffix,
write a function starting with \texttt{Test},
and run \texttt{go\ test\ ./...}.
Both \ac{VSCode} with Go plugin and GoLand
automatically picked up the test setup,
and I was ready to develop with \ac{TDD}.

Standard \texttt{net/http} package~\cite{cox_http_2022} provides everything
needed for setting up \ac{REST} endpoints.
Although I used Gin~\cite{martinez-almeida_gin_2022} for this,
due to a simpler interface,
I used status codes
and \ac{HTTP} client implementation
from \texttt{net/http}.

\paragraph{Refactoring to the standards}\label{sec:refactoring-to-the-standards}

During this project,
I refactored my functions to better suit
the Go standard library.
It was beneficial
not only because of a better integration
with the standard library itself,
but also because of a better integration
with third-party libraries.
There is an unwritten rule,
that every library strives
to be as compatible
with the standard library as possible.

For instance,
in commit \texttt{00547cd}\footfullcite{sewera_chorecoreproducer_2022},
I refactored the \texttt{ToJSON} method
to better suit the standard signatures
(appendix~\ref{apx:method-signature-refactoring-in-go}),
i.e. to return the byte array and error,
just like the \texttt{Marshal} method
from the \texttt{json} package~\cite{cox_json_2022}
in the standard library.

\paragraph{Third-party libraries}\label{sec:third-party-libraries}

Gin was great for writing \ac{REST} endpoints,
with \texttt{gin.Context} having easy access to
standard-library-compatible fields,
making it easily pluggable to other third-party libraries,
like Gorilla WebSocket~\cite{burd_gorilla_2022}.

Zap~\cite{shah_zap_2022} provided a reliable
and performant way to log things in the backend.
Structured logging,
straightforward syntax,
automatic serialization to \ac{JSON} in production mode,
and human-readable format in debug mode,
paired with low or zero-allocation overhead,
made it a perfect choice for logging in a microservice.

\addtocategory{commit}{sewera_chorecoreproducer_2022}

