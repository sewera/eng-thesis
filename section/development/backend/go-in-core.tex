\subsubsection{Go in core}\label{sec:go-in-core}

Go is quickly gaining popularity among developers,
with its great tooling
and a state-of-the-art standard library.
It is an excellent language for writing microservices.
Its focus on this one task
and pragmatism in adding features to the language by its authors,
resulted in an easy to understand and use,
yet very powerful set of tools.

\paragraph{Motivation}\label{sec:motivation}

When choosing the right language for the project,
I focused on finding the right tool
for the application and developer experience.

I wanted \texttt{notipie} to be a high-performance microservice,
so I did not take interpreted languages
like Python or \acf{JS} into consideration.
I mostly considered Java, Kotlin, Rust, and Go.

Java, although popular, does not have the greatest developer experience.
Things like \texttt{equals} and \texttt{hashCode}
are unnecessary bloat in the code.
Project Lombok~\cite{zwitserloot_project_2022} fixes some of them,
but the tooling is limited to IntelliJ,
you have to download a lot of libraries for dealing with \ac{JSON},
create your own code style guide,
and perform a fair bit of setup.

Kotlin, far better than Java,
but also locked-in to IntelliJ with tooling,
was an interesting option for me, but not ideal.

Rust was too low-level for my application.
Explicit memory management, although performant,
was simply too verbose and work-intensive for my use case.

Go was a perfect option.
A plethora of great tooling,
like first-party Go plugin for \ac{VSCode}, GoLand from JetBrains,
community plugins for Neovim,
all working great and providing a good developer experience.
Furthermore, extraordinary performance of the tooling itself,
with tests running in under a second, super-fast compiler,
one of the best standard libraries I have seen,
and overall simplicity of the language, made the choice obvious.
