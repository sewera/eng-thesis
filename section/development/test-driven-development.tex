\subsection{Test-driven Development}\label{sec:test-driven-development}

For the development of the domain,
I used all the best practices I knew.
One of them was test-driven development.
I read about TDD multiple times, in
\citetitle{beck_test-driven_2002}~\cite{beck_test-driven_2002},
\citetitle{martin_clean_2011}~\cite{martin_clean_2011}, and
\citetitle{beck_extreme_2004}~\cite{beck_extreme_2004}.

I followed the red, green, refactor cycle,
checking every time that I understand
the usage of the code I wrote.
It enabled me to have an architecture
in the domain package I am proud of,
without a huge effort.

I only regret that
I did not fully commit to the test-first development.
The \texttt{impl} and \texttt{infra} packages
are not well-tested,
and I even resorted to copying the example implementations
from the Gorilla WebSocket~\cite{burd_gorilla_2022} documentation.
It turned out to be a bad idea,
when I encountered hard to spot bugs,
like the one
where the WebSocket server
was pinging a disconnected client,
resolved in commit \texttt{fc34a8b}~\footfullcite{sewera_fix_2022}.

\addtocategory{commit}{sewera_fix_2022}
