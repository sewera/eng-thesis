\subsection{Command line utility}\label{sec:command-line-utility}

Not everyone who could benefit
from my application
needs to have a good understanding
of Go development.
That is why I also prepared
a command line utility
that is compatible with the most popular shells:
Unix-like shells and Powershell.

Its feature set is very similar
to the one mentioned in the previous section,
but it also includes configuration
and notification templates
for convenience.

\subsubsection{Configuration and notification templates}\label{sec:configuration-and-notification-templates}

This version of a producer can be configured
either through the command line arguments,
or the configuration files,
but the command line arguments
always take precedence.

The configuration files themselves
can be stored in JSON or YAML format.
I chose those two,
because they are very popular formats
for this use case.
Those formats can also be used
for specifying a notification template.

The configuration consists of
the address and port of the backend,
which are set by user.
Also, when the producer sends its first notification
and gets back the App ID,
the producer appends it to the configuration file,
so that it is present in every consecutive push.

The notification template can be any
subset of an App notification model
defined in section~\ref{sec:protocol}.
Everything specified in this template
can be overwritten with CLI arguments.
The only requirement for a valid request
is to have an App name and title set.
The timestamp is set automatically for us.
