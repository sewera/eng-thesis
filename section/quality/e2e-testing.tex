\subsection{E2E Testing}\label{sec:e2e-testing}

There are two ways of E2E testing,
manual and automatic.
I chose the former,
because writing automatic tests
require a complex setup,
a new framework like Selenium~\cite{steward_selenium_2022},
and a whole other project to maintain.
E2E test automation is of substantial advantage
for bigger projects, however,
my project is well tested with
unit tests,
integration tests, and
snapshot tests\footnote{
  Snapshot testing is a frontend-specific term
  explained in section~\ref{sec:ui-testing}.
}.
All of those automatic tests
raised my confidence in the project high enough
that I did not have to manually test
the application very often.

Despite that,
I created two manual testing utilities,
which helped me with development
of frontend and backend components independently,
a test notifications server
and a test WebSocket client.
I chose to write both of them in TypeScript,
with as little code as possible;
first not to introduce another language
to the project, like Python,
and second to reduce the overhead
of a statically-typed language
for such a simple, non-critical task.

\subsubsection{Test Notifications Server}\label{sec:test-notifications-server}

The test notifications server
is a very simple server that sends
randomly-generated notifications to the UI.
It can send an initial batch of notifications
when the UI calls the endpoint synchronously,
but it also can accept a WS connection
and send one notification at a time asynchronously,
on a press of the \textit{Enter} key.

The server itself is written
using Express~\cite{holowaychuck_express_2022},
a lightweight HTTP framework.
The random notification generation
is done with a help of Faker~\cite{marak_faker_2022},
a JS library for generating fake,
but realistically-looking data.

I used it for checking if the UI
is wired up properly,
when I did not have
the notification producer
implementation in place.

\subsubsection{Test WebSocket Client}\label{sec:test-ws-client}

The test WS client
is a simple client that connects
to a hard-coded URL on start,
sets up the WS connection,
and logs everything that is pushed to it.
It also closes the connection before stopping.

For the implementation,
I used the built-in standard WS implementation
in Node~\cite{trott_node_2022}.

I used this test client
to check if the backend properly
sets up the connection,
creates the client object,
sends the data,
and destroys the object properly
after the client disconnects.
