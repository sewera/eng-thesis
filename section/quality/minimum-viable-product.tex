\subsection{Minimum Viable Product}\label{sec:minimum-viable-product}

Eric Ries described the role of the MVP
as a feedback of an idea~\cite{ries_lean_2011}.
Agile Alliance stressed the need
for the quality of that product~\cite{foster_mvp_2022}.
William S. Junk depicted the balance between
the schedule,
resources,
product features,
and quality~\cite{junk_dynamic_2000}.
I knew good quality is a requirement
to present the application to the potential users,
reduce stress,
grow as a professional,
be proud of the work I did,
and prepare for the eventual extension
of the application~\cite{beck_extreme_2004,foster_mvp_2022,martin_clean_2011}.
I wanted my solution to be
as bug-free as possible.

Because of my limited schedule,
resources limited to only one programmer,
and the need for quality,
I did not want to go overboard
with excessive features.

The definition is as follows~\cite{sewera_mvp_2022}:

\begin{quote}
  Notifications can be sent with a manually programmed script
  with the help of a library.
  They arrive to the user interface in real time.

  The library is provided in the MVP.\\
  The backend correctly forwards those notifications to the frontend.\\
  The backend stores existing notifications in memory.\\
  The frontend correctly presents notifications to the user.
\end{quote}
