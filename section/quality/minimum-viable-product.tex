\subsection{Minimum Viable Product}\label{sec:minimum-viable-product}

Eric Ries described the role of the \ac{MVP}
as a reliable feedback of an idea~\cite{ries_lean_2011}.
Agile Alliance stressed the need
for the quality of that product~\cite{foster_mvp_2022}.
William S. Junk characterized the balance between
a schedule,
resources,
product features,
and quality~\cite{junk_dynamic_2000}.

I knew good quality not only is a requirement
to present the application to the potential users,
but also it reduces stress,
allows me to progress as a professional,
be proud of the work I do,
and prepare for the eventual extension
of the application~\cite{beck_extreme_2004,foster_mvp_2022,martin_clean_2011}.
I wanted my solution to be
as bug-free as possible.

Because of my limited schedule,
resources limited to only one programmer,
and the need for quality,
I did not want to incorporate
any excessive features.
The~\acl{MVP} definition is as follows~\cite{sewera_mvp_2022}:

\begin{itemize}
  \item Notifications can be sent with a manually programmed script
        with the help of a library.
  \item They arrive to the \ac{UI} in real time.
  \item The library is provided in the \ac{MVP}.
  \item The backend correctly forwards those notifications to the frontend.
  \item The backend stores existing notifications in memory.
  \item The frontend correctly presents notifications to the user.
\end{itemize}
