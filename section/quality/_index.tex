\section{Quality and testing}\label{sec:quality}

I want to take pride in what I do.
Therefore,
I took extra care when I created Notipie
to account for the quality of the product.
I defined the quality as follows:

\begin{itemize}
  \item sufficient product success;
  \item sufficient product reliability; and
  \item sufficient product maintainability and developer satisfaction.
\end{itemize}

I also defined success as:

\begin{itemize}
  \item sufficient correctness of the delivered information;
  \item sufficient usefulness of the delivered information; and
  \item clear presentation of the information.
\end{itemize}

I did not define success in the commercial terms,
because that was not the scope of my project.
I managed to keep my focus on quality,
and keep the list of features in check
by defining an \ac{MVP}.

Every time I added something to the application,
I ran automated integration test suite
(appendix~\ref{apx:integration-test-core})
in order to check if the application quality
did not deteriorate.
I made sure to run the test suite as often as I could
by making it performant.
And so, one integration test run takes 170 milliseconds in total,
as depicted in listing~\ref{lst:integration-test-run}.
The whole automated test suite ran on every change
pushed to Github servers to ensure the solution quality.
I also double-checked the correct behavior
of my application with manual \ac{E2E} tests.

\subsection{Minimum Viable Product}\label{sec:minimum-viable-product}

Eric Ries described the role of the \ac{MVP}
as a reliable feedback of an idea~\cite{ries_lean_2011}.
Agile Alliance stressed the need
for the quality of that product~\cite{foster_mvp_2022}.
William S. Junk characterized the balance between
a schedule,
resources,
product features,
and quality~\cite{junk_dynamic_2000}.

I knew good quality not only is a requirement
to present the application to the potential users,
but also it reduces stress,
allows me to progress as a professional,
be proud of the work I do,
and prepare for the eventual extension
of the application~\cite{beck_extreme_2004,foster_mvp_2022,martin_clean_2011}.
I wanted my solution to be
as bug-free as possible.

Because of my limited schedule,
resources limited to only one programmer,
and the need for quality,
I did not want to incorporate
any excessive features.
The~\acl{MVP} definition is as follows~\cite{sewera_mvp_2022}:

\begin{itemize}
  \item Notifications can be sent with a manually programmed script
        with the help of a library.
  \item They arrive to the \ac{UI} in real time.
  \item The library is provided in the \ac{MVP}.
  \item The backend correctly forwards those notifications to the frontend.
  \item The backend stores existing notifications in memory.
  \item The frontend correctly presents notifications to the user.
\end{itemize}

\subsection{E2E Testing}\label{sec:e2e-testing}

There are two ways of E2E testing,
manual and automatic.
I chose the former,
because writing automatic tests
require a complex setup,
a new framework like Selenium~\cite{steward_selenium_2022},
and a whole other project to maintain.
E2E test automation is of substantial advantage
for bigger projects, however,
my project is well tested with
unit tests,
integration tests, and
snapshot tests\footnote{
  Snapshot testing is a frontend-specific term
  explained in section~\ref{sec:ui-testing}.
}.
All of those automatic tests
raised my confidence in the project high enough
that I did not have to manually test
the application very often.

Despite that,
I created two manual testing utilities,
which helped me with development
of frontend and backend components independently,
a test notifications server
and a test WebSocket client.
I chose to write both of them in TypeScript,
with as little code as possible;
first not to introduce another language
to the project, like Python,
and second to reduce the overhead
of a statically-typed language
for such a simple, non-critical task.

\subsubsection{Test Notifications Server}\label{sec:test-notifications-server}

The test notifications server
is a very simple server that sends
randomly-generated notifications to the UI.
It can send an initial batch of notifications
when the UI calls the endpoint synchronously,
but it also can accept a WS connection
and send one notification at a time asynchronously,
on a press of the \textit{Enter} key.

The server itself is written
using Express~\cite{holowaychuck_express_2022},
a lightweight HTTP framework.
The random notification generation
is done with a help of Faker~\cite{marak_faker_2022},
a JS library for generating fake,
but realistically-looking data.

I used it for checking if the UI
is wired up properly,
when I did not have
the notification producer
implementation in place.

\subsubsection{Test WebSocket Client}\label{sec:test-ws-client}

The test WS client
is a simple client that connects
to a hard-coded URL on start,
sets up the WS connection,
and logs everything that is pushed to it.
It also closes the connection before stopping.

For the implementation,
I used the built-in standard WS implementation
in Node~\cite{trott_node_2022}.

I used this test client
to check if the backend properly
sets up the connection,
creates the client object,
sends the data,
and destroys the object properly
after the client disconnects.

\subsection{Code quality}\label{sec:code-quality}

Apart from the quality of the product,
or rather in concert with it
comes the code quality.
I spent a fair amount of time trying
to perfect maintainability,
scalability,
and reusability of the code.
One big thing that helped me with this task
was test-first development,
explained many times by Kent Beck or
Robert C. Martin~\cite{beck_extreme_2004,beck_test-driven_2002,martin_clean_2011}.
I cannot stress enough how big
of a change in development pace it brought.
Another thing was refactoring,
explained in detail,
with a handy glossary,
by Martin Fowler~\cite{fowler_refactoring_2019}.
Together with \ac{XP}~\cite{beck_extreme_2004},
it ensured me that designing the architecture right
the first time rarely works.

\citetitle{millett_patterns_2015}~\cite{millett_patterns_2015}
and \citetitle{evans_domain-driven_2003}~\cite{evans_domain-driven_2003}
showed me a different perspective
of not getting something right the first time,
but this time it was specifically the domain.
For instance,
I struggled a lot with the Tag definition,
name, and set of responsibilities.
The informal discussions with my friends
helped me gain additional insight into the domain
from a different point of view.

Those two books also broke down the intricacies
of writing the code for reuse or replacement.
I could, for example,
migrate pretty easily
from one state management implementation to another,
as described in section~\ref{sec:state-management-in-ui}.
I can also easily swap
the persistence implementation in the backend,
from in-memory to a proper database.

