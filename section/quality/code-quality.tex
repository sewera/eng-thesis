\subsection{Code quality}\label{sec:code-quality}

Apart from the quality of the product,
or rather in concert with it
comes the code quality.
I spent a fair amount of time trying
to perfect maintainability,
scalability,
and reusability of the code.
One big thing that helped me with this task
was test-first development,
explained many times by Kent Beck or
Robert C. Martin~\cite{beck_extreme_2004,beck_test-driven_2002,martin_clean_2011}.
I cannot stress enough how big
of a change in development pace it brought.
Another thing was refactoring,
explained in detail,
with a handy glossary,
by Martin Fowler~\cite{fowler_refactoring_2019}.
Together with \ac{XP}~\cite{beck_extreme_2004},
it ensured me that designing the architecture right
the first time rarely works.

\citetitle{millett_patterns_2015}~\cite{millett_patterns_2015}
and \citetitle{evans_domain-driven_2003}~\cite{evans_domain-driven_2003}
showed me a different perspective
of not getting something right the first time,
but this time it was specifically the domain.
For instance,
I struggled a lot with the Tag definition,
name, and set of responsibilities.
The informal discussions with my friends
helped me gain additional insight into the domain
from a different point of view.

Those two books also broke down the intricacies
of writing the code for reuse or replacement.
I could, for example,
migrate pretty easily
from one state management implementation to another,
as described in section~\ref{sec:state-management-in-ui}.
I can also easily swap
the persistence implementation in the backend,
from in-memory to a proper database.
