\section{Introduction}\label{sec:introduction}

We get notifications every day.
Unfortunately,
those notifications either arrive
to our e-mail inboxes,
or worse,
we have to manually check
each and every website an event may have occurred on.

I personally use \ac{CI} -- \ac{CD} pipelines
on Github Actions for my private projects,
and Jenkins for work.
I get loads of e-mail notifications from Icinga.
I check production servers' state on Grafana,
which also sends notifications
for severe events to my work Microsoft Teams.
I check the health of the services
running on my private server\footnote{
  An example of this particular use case
  is presented in appendix~\ref{apx:sample-systemd-service-with-notipie-hooks}.
} with a custom shell script.
The number of notifications I get every day
is usually too big to wrap my head around,
let alone the fact,
that they come from at least five different sources.

Notipie is a response to this problem.
It is specifically designed
to handle custom sources of notifications,
provide easy setup,
good performance,
and a simple interface.
The project's source code can be found
on Github: \url{https://github.com/blazejsewera/notipie}.

In this thesis,
I will demonstrate my solution
by first explaining the problem domain in detail,
and providing necessary definitions.
I will then quickly go through
the project management
to show how this project was conducted,
as well as which tools and practices
I used to achieve my goal.
The next section
will outline the process of implementing
each component in the system,
describe the protocol for communication
between those components,
and talk about the programming practices I used.
I will then describe how I ensured
a good product quality,
and my approach to testing.
After that,
I will lay out the takeaways from the development
of this project.
I will show what went right
and what I wish I had done differently.

\subsection{Existing Solutions}\label{sec:existing-solutions}

There are many solutions
for systems monitoring,
there are e-mail filters
and smart directories.
Most modern operating systems have some kind of a notification center.
So why did I make Notipie in the first place?

First of all,
I needed something \emph{simple}.
I didn't want the hassle of setting up
the whole Grafana stack for my simple use case,
I also didn't want to go through the trouble
of setting an e-mail gate
for my server events and health checks.
I~also~considered IFTTT,
but the pricing~\cite{ifttt_plans_2022}
was too much for my simple usage,
and the built-in integrations
were not suited for my notification needs.

When I thought about easy-to-use,
standardized ways of displaying notifications,
I immediately looked at
the notification systems on smartphones.
Android and iOS have very simple,
yet very usable notifications,
that all go into a single place.
I wanted to have similar behavior
for my custom use cases,
without the need to~install any apps
from the Apple AppStore,
or Google Play Store.

