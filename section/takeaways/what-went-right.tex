\subsection{What went right}\label{sec:what-went-right}

The thing I am most satisfied with
is the lack of the database.
It was not dictated by my reluctancy to add one,
but rather an informed decision
to postpone the whole database setup,
and not to include it in an \ac{MVP}.

I set up a \ac{CI} pipeline
in Github Actions~\cite{github_inc_github_2022-1}
pretty early in the project,
before I even read how important it is
from Kent Beck~\cite{beck_extreme_2004}.
It really sped up my development cycle,
and provided me with invaluable feedback,
even when I pushed the code and turned off my computer.

As I was writing the core domain code,
I followed another Kent Beck's advice,
and I left breaking tests between commits.
It was an advice for single-developer projects
utilizing \ac{TDD}~\cite{beck_test-driven_2002}.
When there are at least two developers,
it is much easier to keep track of what is going on.
When I get distracted,
I lose focus and current context of the code.
In pair programming,
the second programmer can take over this context
and maintain continuity.
In a single-developer project,
the failing tests act as the second programmer
in this regard,
and they mark a clear place
to take up where I left off.

I am very proud of my experiments
on the code,
like Reactive Raven~\cite{sewera_reactive_2022},
explained in section~\ref{sec:reactive-raven},
or connecting snapshot testing
with a \ac{CSS} utility library
to avoid slow visual tests,
which is explained in section~\ref{sec:ui-testing}.
