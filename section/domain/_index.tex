\section{Domain}\label{sec:domain}

\Ac{DDD} is a software design approach
that aligns the code with the reality
of the problem domain.
It achieves said alignment by focusing
on the following components of the problem domain:
its terminology,
the core reasons behind why the software is being developed,
and success definition.
Because of this alignment,
adding new features is easy,
and the understanding of the problem domain
stays in sync with the production code~\cite{millett_patterns_2015}.
A model of the solution is called the domain model.
The production code needs to directly reflect the domain model,
which means that \iac{UL}, a set of definitions
understandable both to domain experts
and to developers~\cite{evans_domain-driven_2003,millett_patterns_2015},
has to be defined.
\Ac{UL} helps with confronting the proposed solutions
between these two groups.

I started tackling the problem domain by
laying out the high-level definitions
of all significant parts
that are necessary to implement the solution.
I wanted the domain of my application
to be as simple and clear as possible.
Therefore,
I came up with \iac{UL}
and defined four main components: Notification, Tag, App, and User.

\subsection{Notification}\label{sec:notification}

The core of the application.
This is the structure
acting as a protocol of communication
between all components in the application domain.
It consists of the text fields,
like title, optional subtitle, optional body,
\acp{URI} for marking the Notification read,
but also an App\footnote{
  The app component is explained in detail in section~\ref{sec:app}
} of origin.

For the distributed nature of the components,
the Notification is always passed by value.
Having an immutable copy of the Notification object
in every place ensures
no unexpected states in the application.
It is easily achieved
using the features of
the Go language\footnote{
  More on Go features in section~\ref{sec:the-benefits-of-using-go}
}.

\subsection{Tag}\label{sec:tag}

The Tag in the Notipie domain
is a Notification broker
between the Apps and the Users.
It stores which Apps are attached
and which Users are subscribed to it.
It also ensures thread-safety
for attaching and detaching of Apps,
as well as subscribing and unsubscribing of Users.

At first,
I wanted it to be named ``Room'',
like a chat room,
so that many Users could have been
in different Rooms,
similar to subscribing to multiple Tags.
However,
it was less understood by people I spoke with,
mainly due to the fact,
that one App sending to different rooms,
and a User being in different rooms simultaneously,
contradicts the existing understanding of chat rooms.

A User can subscribe to the Tag,
and an App can add itself to a Tag.
When that happens,
the forwarding of Notifications
between said App and User is enabled.
Thread-safety of those operations
are ensured by mutexes\footnote{
  A mutual exclusion lock.
  A mutex serializes the execution
  of multiple threads~\cite{mattson_patterns_2004}.
}.
The Tag is actively listening
on the Notification channel (\texttt{NotificationChan}),
and every time a new Notification arrives,
the \texttt{broadcast} method~(listing~\ref{lst:broadcast-method-in-tag})
is executed.
The \texttt{broadcast} method then sends that Notification
to every User subscribed to this Tag.
The flow is presented in figure~\ref{fig:notification-flowchart}.

\begin{figure}[h]
  \centering
  \includegraphics[width=10cm,keepaspectratio]{chart/out/notification-flowchart.pdf}
  \caption{Notification flow in \texttt{domain}}
  \label{fig:notification-flowchart}
\end{figure}

\subsubsection{Technicalities}\label{sec:tag-technicalities}

There is an obvious problem with this approach.
When a User is subscribed to two Tags,
and one App has those two Tags assigned,
then that User can get the same Notification twice
(figure~\ref{fig:duplicated-notification}).
There are two solutions I could think of.

\begin{figure}[h]
  \centering
  \includegraphics[width=10cm,keepaspectratio]{chart/out/duplicated-notification.pdf}
  \caption{Duplicated Notification flow}
  \label{fig:duplicated-notification}
\end{figure}

\paragraph*{Solution 1}\label{par:duplication-solution-1}

This solution assumes the following:

There is a new component in the domain, Router.
It is a centralized component
that knows which App is connected to which Tag,
and which User is subscribed to those Tags
(figure~\ref{fig:notification-router}).

\begin{figure}[h]
  \centering
  \includegraphics[width=15.5cm,keepaspectratio]{chart/out/notification-router.pdf}
  \caption{Notification Router}
  \label{fig:notification-router}
\end{figure}

With this approach,
the User component is simpler,
as it does not need any deduplication logic.
However, there is a whole new component
in the domain, which introduces complexity.
Furthermore, the Router has to be
a Singleton~\cite{gamma_design_1994}
in order to process all the Notifications.
It would create a bottleneck,
as the Router would need to process every new Notification
that goes through the application.

The Router needs a table referencing
Tags and Users subscribed to them.
Provided that there are $n$ Tags,
and every Tag has $m$ Users subscribed to it,
the lookup complexity of such table
would be $O(n \cdot m)$,
because the table would need to be
scanned in its entirety to determine
all the Users that need to get the Notification.
It could be optimized into $O(m \cdot \log n)$
if the Tag list is sorted,
but the sorting would introduce
more complexity in the Router.

The time complexity was not the main concern, though.
Martin Fowler insisted on doing performance optimizations
after taking care of code complexity~\cite{fowler_refactoring_2019},
so my main concern was additional code complexity,
compared to the second solution.

\paragraph*{Solution 2}\label{par:duplication-solution-2}

This solution assumes the following:

The deduplication logic is inside the User,
every Notification is sent by a Tag to a User
if the User is subscribed to the Tag.
The User decides if it keeps the Notification
or drops it as a duplicate
(figure~\ref{fig:notification-user-dedupe}).

\begin{figure}[h]
  \centering
  \includegraphics[width=15.5cm,keepaspectratio]{chart/out/notification-user-dedupe.pdf}
  \caption{Notification Deduplication in User}
  \label{fig:notification-user-dedupe}
\end{figure}

With this approach,
the User component is more complex,
but the overall domain structure is simpler.
The Users are distributed across the application,
and only need to deduplicate their own Notifications.

The solution is parallelized by nature,
so there is no bottleneck
like with the previous example.
The solution is also more scalable.
As the application grows,
and more Notifications go through it,
we can start to observe race conditions.
In result,
duplicated messages from different Tags
will start to appear.

Then, we can simply extend the length of
previous Notification \acp{ID} that we check.
We can start with keeping as many previous
Notification \acp{ID}
as there are Tags the User is subscribed to.
The code complexity is also superior
to the previous approach.
Dropping or saving a Notification is a trivial task,
with a simple algorithm:
when the current Notification \ac{ID} is found in the
already received \acp{ID}, drop it; otherwise, save it.

\subsection{App}\label{sec:app}

The App is able to send Notifications
in the domain.
Conceptually,
there is one instance of an App
per each real application
able to produce notifications.
Multiple Apps per one producer scenario
is explained in detail in section~\ref{sec:producer-usage}.

For instance,
when I set up a health checking script
that produces a notification each time
a Systemd service is started,
like the one in listing~\ref{lst:sample-systemd-service-definition},
when any of the hooks fire,
and the notification arrives to the backend,
a new App is created in the domain.
When this happens,
a new App ID is generated
and is later sent back to the producer.

\subsection{User}\label{sec:user}

The User is a recipient of the Notifications.
Whenever a Notification
goes through the Tag a User is subscribed to,
they get this Notification.
The User is also responsible for
the deduplication of the Notifications,
should they arrive from multiple different Tags.
It is explained in detail in section~\ref{sec:tag-technicalities}.

