\subsection{Tag}\label{sec:tag}

The Tag in the Notipie domain
is a Notification broker
between the Apps and the Users.
It stores which Apps are attached to it
and which Users are subscribed to it.
It also ensures thread-safety
for attaching and detaching of Apps,
as well as subscribing and unsubscribing of Users.

At first,
I wanted it to be named a Room,
like a chat room,
so that many Users could have been
in different Rooms,
similar to subscribing to multiple Tags.
However,
it was less understood by people I spoke with,
mainly due to the fact,
that one App sending to different rooms,
and a User being in different rooms simultaneously
contradicts the existing understanding of chat rooms.

A User can subscribe to the Tag,
and an App can add itself to a Tag.
The thread-safety of those operations
are ensured by mutexes.
When that happens,
the forwarding of Notifications
between said App and User is enabled.
The Tag is actively listening
on the Notification channel (\texttt{NotificationChan}),
and every time a new Notification arrives,
the \texttt{broadcast} method~(listing~\ref{lst:broadcast-method-in-tag})
is executed.
The \texttt{broadcast} method in turn sends that Notification
to every User subscribed to this Tag.
The flow is presented in figure~\ref{fig:notification-flowchart}.

\begin{figure}[h]
  \centering
  \includegraphics[width=10cm,keepaspectratio]{chart/out/notification-flowchart.pdf}
  \caption{Notification flow in \texttt{domain}}
  \label{fig:notification-flowchart}
\end{figure}

\subsubsection{Technicalities}\label{sec:tag-technicalities}

There is an obvious problem with this approach.
When a User is subscribed to two Tags,
and one App has those two Tags assigned,
then that User can get the same Notification twice
(figure~\ref{fig:duplicated-notification}).
There are two solutions I could think of.

\begin{figure}[h]
  \centering
  \includegraphics[width=10cm,keepaspectratio]{chart/out/duplicated-notification.pdf}
  \caption{Duplicated Notification flow}
  \label{fig:duplicated-notification}
\end{figure}

\paragraph*{Solution 1}\label{par:duplication-solution-1}

This solution assumes the following:

There is a new component in the domain, Router.
It is a centralized component
that knows which App is connected to which Tag,
and which User is subscribed to those Tags
(figure~\ref{fig:notification-router}).

\begin{figure}[h]
  \centering
  \includegraphics[width=15.5cm,keepaspectratio]{chart/out/notification-router.pdf}
  \caption{Notification Router}
  \label{fig:notification-router}
\end{figure}

With this approach,
the User component is simpler,
as it does not need any deduplication logic.
However, there is a whole new component
in the domain, which introduces complexity.
Furthermore, the Router has to be
a Singleton~\cite[pp.~127-134]{gamma_design_1994}
in order to process all the Notifications.
It would create a bottleneck,
as the Router would need to process every new Notification
that goes through the application.

The Router needs a table referencing
Tags and Users subscribed to them.
Provided that there are $n$ Tags,
and every Tag has $m$ Users subscribed to it,
the lookup complexity of such table
would be $O(n \cdot m)$,
because the table would need to be
scanned in its entirety to determine
all the Users that need to get the Notification.
It could be optimized into $O(m \cdot \log n)$
if the Tag list is sorted,
but the sorting would introduce
more complexity in the Router.

The time complexity was not the main concern, though.
Martin Fowler insisted on doing performance optimizations
after taking care of code complexity~\cite{fowler_refactoring_2019},
so my main concern was additional code complexity,
compared to the second solution.

\paragraph*{Solution 2}\label{par:duplication-solution-2}

This solution assumes the following:

The deduplication logic is inside the User,
every Notification is sent by a Tag to a User
if the User is subscribed to the Tag.
The User decides if it keeps the Notification
or drops it as a duplicate
(figure~\ref{fig:notification-user-dedupe}).

\begin{figure}[h]
  \centering
  \includegraphics[width=15.5cm,keepaspectratio]{chart/out/notification-user-dedupe.pdf}
  \caption{Notification Deduplication in User}
  \label{fig:notification-user-dedupe}
\end{figure}

With this approach,
the User component is more complex,
but the overall domain structure is simpler.
The Users are distributed across the application,
and only need to deduplicate their own Notifications.

The solution is parallelized by nature,
so there is no bottleneck
like with the previous example.
The solution is also more scalable.
As the application grows,
and more Notifications go through it,
we can start to observe race conditions.
In result,
duplicated messages from different Tags
will start to appear.

Then, we can simply extend the length of
previous Notification IDs that we check.
In the worst case,
we would have to keep as many previous
Notification IDs,
as there are Tags the User is subscribed to.
The code complexity is also superior
to the previous approach.
Dropping or saving a Notification is a trivial task,
with a simple algorithm:
when the current Notification ID is found in the
already received IDs, drop it; otherwise, save it.
