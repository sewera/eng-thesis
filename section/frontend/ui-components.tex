\subsection{UI components}\label{sec:ui-components}

Using components enabled me to split my UI
into small, reusable components,
eliminating code duplication,
and helping with maintaining the consistent look.
I wanted to choose the right library
for this task,
so that the development would be quick,
and I would have plenty of tools
that would help me achieve
good architecture for the frontend.

\subsubsection{UI component library}\label{sec:ui-component-library}

When choosing the library for the UI components, I considered:

\begin{itemize}
  \item
        React~\cite{oshannessy_react_2022},
  \item
        Vue.js~\cite{you_vuejs_2022}, and
  \item
        Angular~\cite{kalpakas_angular_2022}.
\end{itemize}

All those libraries are very popular,
so I chose React,
because I had the most experience with it in my professional work.

\subsubsection{The component directory}\label{sec:the-component-directory}

All of the frontend components
are located in the component directory~\cite{sewera_notipie_2022-3}.
The directories are divided by category.
There are currently two categories,
\emph{canvas}, and
\emph{notification}.

Canvas category includes
an \emph{AppCanvas} component,
responsible for displaying
a background in the right color,
and controlling the light or dark mode setting.

Notification category includes
a notification card\footnote{
  The design of the card
  is explained in detail
  in section~\ref{sec:final-design}.
}, a notification container,
which contains all the notifications
from the same App\footnote{
  The concept of an App is described
  in section~\ref{sec:app}.
}, and a notification board,
containing all the notifications
presented in the UI.

\subsubsection{Testing}\label{sec:ui-testing}



\subsubsection{Storybook}\label{sec:storybook}
