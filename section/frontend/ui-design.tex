\subsection{UI Design}\label{ui-design}

When designing the UI of Notipie, I tried to maximize usability, and
minimize complexity of the interface.

\subsubsection{Inspirations}\label{inspirations}

My main inspirations for the interface were Apple Human Interface
Guidelines\footnote{Apple Human Interface Guidelines, retrieved
2022-05-31.
\url{https://developer.apple.com/design/human-interface-guidelines/}}
and Google's Material Design\footnote{Google's Material Design,
retrieved 2022-05-31. \url{https://material.io}}, but by far the most
inspiration was taken from Github Primer\footnote{Github Primer,
retrieved 2022-05-31. \url{https://primer.style}}.

I tried to break down what is useful, what is unnecessary in my project,
and extract only the essentials for my design.

\subsubsection{Final design}\label{final-design}

\paragraph{The card}\label{the-card}

The card is a building block for the entire user interface. It provides
the most interaction in the whole application, therefore it had to be
designed with clearly laid out information and intuitive controls.

\begin{figure}
  \centering
  \includegraphics[width=8cm,keepaspectratio]{./img/card_labeled.png}
  \caption{The card with labeled elements}
\end{figure}

The card itself consists of several elements:

\begin{enumerate}
  \item
    logo, it can be an image or automatically generated SVG from the first
    two letters of the app's name,
  \item
    indicator, whether the notification has been seen or not,
  \item
    title of the notification,
  \item
    subtitle,
  \item
    body, that collapses after it reaches a certain length, so that an
    ellipsis appears (\texttt{...}),
  \item
    information about what app sent the notification and when it happened,
  \item
    controls to archive, mark as read, or go to external site connected
    with the notification, like a certain build on Jenkins, or the
    notification page on Github.
\end{enumerate}

\begin{figure}
  \centering
  \includegraphics[width=8cm,keepaspectratio]{./img/card_guides.png}
  \caption{The card with guides}
\end{figure}

The card was also designed with aesthetics in mind. All elements were
carefully positioned and aligned, so they are not only pleasant to look
at, but also have some features important for visual communication:

\begin{itemize}
  \item
    the rounded corners take the focus away from the card frame, and
    provide a natural, neutral enclosure for the notification,
  \item
    the inner padding is of equal size in each direction to provide
    optical stability,
  \item
    the distance between the logo and title -- subtitle combo is the same
    size as the padding, making the logo appear centered,
  \item
    the title -- subtitle combo itself is centered vertically relative to
    the logo,
  \item
    the distances between the logo, notification body, and app name --
    timestamp combo are shorter in order to make the inner section more
    connected,
  \item
    the controls are centered relative to the app name -- timestamp combo,
  \item
    the \emph{unread} indicator is unobtrusive enough not to steal all the
    focus from the card's content,
  \item
    finally, the \emph{unread} indicator is positioned slightly outside
    the inner section, so that it belongs to the card itself, not its
    content, therefore it is easier to spot at a glance.
\end{itemize}
