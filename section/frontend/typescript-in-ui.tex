\subsection{TypeScript in UI}\label{sec:typescript-in-ui}

I decided to use TypeScript in my project for the frontend part,
because of its type checking tools,
huge popularity,
and a growing demand for in on the job market.

\subsubsection{Choosing the language}\label{sec:choosing-the-language}

When choosing which language to use in the UI,
I considered a couple of options:

\begin{itemize}
      \item
            plain JavaScript,
      \item
            TypeScript,
      \item
            Elm, and
      \item
            CoffeeScript.
\end{itemize}

I immediately discarded the last two,
due to their smaller popularity,
compared to JavaScript or TypeScript.

The featureset of the language was also very important to me.
JavaScript is by far the most popular,
but it lacks type annotations or pre-runtime type checking.
TypeScript and Elm turned out to be winners in the type checking toolchain.

TypeScript also has a big advantage of being very similar to plain JavaScript,
so the transpiled code is very readable.

A big factor was general trend of language's popularity growth.
TypeScript was a clear winner in this scenario,
being third most loved language
and second most wanted language
in the Stack Overflow Developer Survey 2021~\cite{stack_overflow_2021_2021}.

It was only beaten by Rust and Clojure
in the \emph{Most Loved} section,
both of which are non-frontend languages,
and Python in the \emph{Most Wanted} section,
which is also not a frontend language.

Another report confirming the growing popularity of TypeScript
is Github Octoverse Report 2021~\cite{github_inc_2021_2021}.
Since 2017,
it beat
Ruby,
C,
C++,
C\#,
Shell, and
PHP
and is, as of 2021, the fourth top language on Github.

\subsubsection{Working with TypeScript}\label{sec:working-with-typescript}

Starting with TypeScript was fairly easy.
The toolchain was included in the project creation scripts.
Most dependencies had good TypeScript annotations,
or they were completely written in TypeScript,
which was very helpful for maintaining type safety.

Learning the language was also very easy.
I was already familiar with JavaScript,
so I only needed to learn the type annotations,
which were very intuitive to use.
