\subsection{Build system}\label{build-system}

For the build and bundle software, I wanted to use something modern,
with hot module reloading, easy to use setup scripts, customizable
development server, and short bundle times.

\subsubsection{Snowpack and Vite}\label{snowpack-and-vite}

I started with Snowpack\footnote{Snowpack, retrieved 2022-05-31.
\url{https://www.snowpack.dev}} and used it until I decided to move to
Vite\footnote{Vite, retrieved 2022-05-31. \url{https://vitejs.dev}} in
commit \texttt{c11bc35}\footnote{commit \texttt{c11bc35}, retrieved
2022-05-31.
\url{https://github.com/blazejsewera/notipie/commit/c11bc35370f512f35d522a55fcd216c1c80ea75a}}.

Snowpack offered both hot module reloading and short bundle times,
however, there were some minor problems from time to time, the project
had a slow development, and the alternative, Vite did not seem to have
those problems.

I tried Vite in my other project, Reactive Raven\footnote{Reactive
Raven, retrieved 2022-05-31.
\url{https://github.com/blazejsewera/reactive-raven}}, and the
integration with React, TypeScript, Tailwind CSS, and other tools I used
was seamless, therefore I decided to migrate to it in Notipie as well.

On April 20th, 2022, Snowpack's maintainer stated in the project's
Readme document (commit \texttt{45456aa})\footnote{project's Readme
document (commit \texttt{45456aa}), retrieved 2022-05-31.
\url{https://github.com/FredKSchott/snowpack/blob/45456aa14978460afcb5ce20f7296556d22c7595/README.md}}
that they would no longer maintain the project and mentioned Vite as a
good alternative for it.
