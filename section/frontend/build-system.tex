\subsection{Build system}\label{build-system}

For the build and bundle software,
I wanted to use something modern,
with hot module reloading,
easy to use setup scripts,
customizable development server,
and short bundle times.

\subsubsection{Snowpack and Vite}\label{snowpack-and-vite}

I started with Snowpack~\cite{schott_snowpack_2021}
and used it until I decided to move to Vite~\cite{you_vite_2022}
in commit \texttt{c11bc35}\footnote{
  B. Sewera,
  \textit{``chore(ui): move to vite instead of snowpack'': Commit}
  \texttt{c11bc35}
  \textit{in Notipie},
  2021.
    [Online]. (visited on 2022-05-31).
  \url{https://github.com/blazejsewera/notipie/commit/c11bc35370f512f35d522a55fcd216c1c80ea75a}
}.

Snowpack offered both hot module reloading and short bundle times.
However, there were some minor problems from time to time,
the project had a slow development,
and the alternative,
Vite did not seem to have those problems.

I tried Vite in my other project,
Reactive Raven~\cite{sewera_reactive_2022},
and the integration with
React,
TypeScript,
Tailwind CSS,
and other tools I used was seamless,
therefore I decided to migrate to it in Notipie as well.

On April 20th, 2022,
Snowpack's maintainer stated in the project's Readme document
(commit \texttt{45456aa}\footnote{
  F. K. Schott,
  \textit{``README.md'', as of commit} \texttt{45456aa} \textit{in Snowpack}, 2022.
    [Online]. (visited on 2022-05-31).
  \url{https://github.com/FredKSchott/snowpack/blob/45456aa14978460afcb5ce20f7296556d22c7595/README.md}
})
that he would no longer maintain the project,
and mentioned Vite as a good alternative for it.
