\subsection{Go in core}\label{sec:go-in-core}

Go is quickly gaining popularity among developers,
with its great tooling, focus on microservices,
and state-of-the-art standard library.

\subsubsection{Motivation}\label{sec:motivation}

When choosing the right language for the project,
I focused on finding the right tool
for the application and developer experience.

I wanted \texttt{notipie} to be a high-performance microservice,
so I did not take interpreted languages
like Python or JavaScript into consideration.
I mostly considered Java, Kotlin, Rust, and Go.

Java, although popular, does not have the greatest developer experience.
Things like \texttt{equals} and \texttt{hashCode}
are unnecessary bloat in the code.
Project Lombok~\cite{zwitserloot_project_2022} fixes some of them,
but the tooling is limited to IntelliJ,
you have to download a lot of libraries for dealing with JSON,
create your own code style guide,
and perform a fair bit of setup.

Kotlin, far better than Java,
but also locked-in to IntelliJ with tooling,
was an interesting option for me, but not ideal.

Rust was too low-level for my application.
Explicit memory management, although performant,
was simply too verbose and work-intensive for my use case.

Go was a perfect option.
A plethora of great tooling,
like first-party Go plugin for VSCode, GoLand from JetBrains,
community plugins for Neovim,
all working great and providing a good developer experience.
Furthermore, extraordinary performance of the tooling itself,
with tests running in under a second, super-fast compiler,
one of the best standard libraries I have seen,
and overall simplicity of the language, made the choice obvious.

\subsubsection{How did Go make the development easier}\label{sec:how-did-go-make-the-development-easier}

\paragraph*{Built-in language features}\label{par:built-in-language-features}

The features that helped the most during development
were channels and \emph{goroutines},
coroutines automatically managed by the Go runtime.
The idea behind those was very simple to understand,
and working with concurrent programming was a lot easier.

\paragraph*{Standard library}\label{par:standard-library}

Standard \texttt{testing} package~\cite{cox_testing_2022} provided
a unified, and simple tooling for testing.
I did not have to think anything about test setup.
No custom scripts, third-party libraries, or IDE setup.
All I needed to do was to name a file with a \texttt{\_test.go} suffix,
write a function starting with \texttt{Test},
and run \texttt{go\ test\ ./...}.
Both VSCode with Go plugin and GoLand
automatically picked up the test setup,
and I was ready to develop with TDD.

Standard \texttt{net/http} package~\cite{cox_http_2022} provides everything
needed for setting up REST endpoints.
Although I used Gin~\cite{martinez-almeida_gin_2022} for this,
due to a simpler interface,
I used status codes and HTTP client implementation from \texttt{net/http}.

\paragraph*{Third-party libraries}\label{par:third-party-libraries}

Gin was great for writing REST endpoints,
with \texttt{gin.Context} having easy access to
standard-library-compatible fields,
making it easily pluggable to other third-party libraries,
like Gorilla WebSocket~\cite{burd_gorilla_2022}.

Zap~\cite{shah_zap_2022} provided a reliable
and performant way to log things in the backend.
Structured logging,
straightforward syntax,
automatic serialization to JSON in production mode,
and human-readable format in debug mode,
paired with low or zero-allocation overhead,
made it a perfect choice for logging in a microservice.
