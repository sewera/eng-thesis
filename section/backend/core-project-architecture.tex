\subsection{Core project architecture}\label{sec:core-project-architecture}

The project on the top-level is structured over 4 directories:

\begin{itemize}
      \item
            \texttt{cmd} -- entry point to the application (\texttt{main})
      \item
            \texttt{internal} -- application-specific code
      \item
            \texttt{pkg} -- reusable utils, not specific to the application
      \item
            \texttt{test} -- black box integration test code
\end{itemize}

The directory names are taken
from the \citetitle{quest_standard_2022}~\cite{quest_standard_2022}.

\subsubsection{The internal directory}\label{sec:the-internal-directory}

Application-specific code is split into 5 directories
representing the levels of abstraction:

\begin{itemize}
      \item
            \texttt{domain} -- business logic,
            defines data structures and communication of domain objects
            on the highest level of abstraction,
      \item
            \texttt{grid} -- lower level of abstraction than domain,
            defines proxies that convert network models into domain models,
            creates and organizes domain objects into a grid,
            in which those objects can communicate,
      \item
            \texttt{impl} -- implements network endpoints,
            WebSockets, and persistence,
      \item
            \texttt{infra} -- configures the application
            and sets up the context for DI.
\end{itemize}

There also was a \texttt{model} package
in the \texttt{internal} directory,
but I moved it to an exportable \texttt{pkg} directory,
to reuse it in the notification producer.

\subsubsection{The Grid}\label{sec:the-grid}

TODO: describe the grid
